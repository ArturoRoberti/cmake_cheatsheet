% pybind with examples
\vspace{2cm}
\section{Pybind11} \label{s:pybind11}
Used to create a python package out of a C++ library and requires \texttt{CMake}, \texttt{C++} and a \texttt{pyproject.toml} file.

\subsection{CMake}
The CMake implementation example is given in the main example. The main steps include:
\begin{itemize}
    \item \cmakebox{set_target_properties(example_library PROPERTIES POSITION_INDEPENDENT_CODE ON)} - Necessary, as the later built dynamic library (\bashbox{.so}-file) will not have a permanent memory address
    \item \cmakebox{pybind11_add_module(example_py MODULE ${PYBIND_SRC})} - Custom Pybind command defining the new python module
    \item \cmakebox{target_link_libraries(example_py PRIVATE example_library)} - Link the custom library to the python module
\end{itemize}

Furthermore, the \texttt{pybind11.cmake} and \texttt{pybind11\_download.cmake} files are similar to their \textit{eigen} equivalents (Sections \ref{ss:auto_installation/eigen.cmake}, \ref{ss:auto_installation/eigen_download.cmake}).
Additionally, \textit{pybind} is linked to the main library via \cmakebox{add_subdirectory(${PROJECT_SOURCE_DIR}/externals/pybind/pybind11 EXCLUDE_FROM_ALL)}.

\subsection{C++}
Below an example of a \texttt{pybind\_wrapper.cpp} file used in CMake to create a python module:
\begin{lstlisting}[style=CppStyle]
    #include <pybind11/eigen.h>
    #include <pybind11/pybind11.h>
    #include <sstream>

    #include "example.hpp"

    namespace py = pybind11;
    using namespace example;

    PYBIND11_MODULE(example_py, m) {
        py::class_<ExampleClass>(m, "ExampleClassPy")
            // Constructor(s)
            .def(py::init<>())

            // Singular methods
            .def("getSize", &ExampleClass::getSize)
            .def("getVector", &ExampleClass::getVector)

            // Overloaded method  - No input
            .def("printClassVector",
                 static_cast<void (ExampleClass::*)()>
                 (&ExampleClass::printVector))

            // Overloaded method - Vector input
            .def("printVector",
                 static_cast<void (ExampleClass::*)(const Eigen::VectorXd&)>
                 (&ExampleClass::printVector));

        // Further classes and functions...
    }
\end{lstlisting}
For further information, see the \href{https://pybind11.readthedocs.io/en/stable/}{pybind documentation}.

\subsection{Pip installation}
A pyproject.toml file is necessary to install the python package via pip. An example is given below:
\begin{lstlisting}[style=CmakeStyle]
    [build-system]
    requires = ["setuptools", "wheel"]
    build-backend = "setuptools.build_meta"

    [project]
    name = "example_py"
    version = "0.1.0"
    description = "Python bindings for the C++ example module"
    authors = [{ name = "Arturo Roberti" }]
    readme = "README.md"

    [tool.setuptools]
    packages = []
    include-package-data = true

    [tool.setuptools.package-data]
    example_py = ["build/example_py*.so"]

    [tool.setuptools.dynamic]
    dependencies = { file = ["requirements.txt"] }  # Optional
\end{lstlisting}

This file is best added to the main directory and not the build directory to avoid accidental deletion.
Then, the package can be installed from the main directory via \bashbox{pip install -e .} (\textit{editable mode} flag \texttt{-e} optional). However, as the shared library (\bashbox{.so}-file) is built in the \textit{build} directory, the \texttt{PYTHONPATH} variable must be extended to include the build directory.
This can be done by adding the following line to the \texttt{.bashrc} file:
\begin{lstlisting}[style=BashStyle]
    export PYTHONPATH=$PYTHONPATH:/path/to/build
\end{lstlisting}