\section{Variables} \label{s:variables}



\subsection{Base Types} \label{ss:variables/base_types}
The existing variable types are:
\begin{itemize}
    \item \cmakebox{STRING} - Generic string value
    \item \cmakebox{BOOL} - Boolean ON/OFF value
    \item \cmakebox{PATH} - Path to a directory
    \item \cmakebox{FILEPATH} - Path to a file
    \item \cmakebox{INTERNAL} - Do not present in GUI at all
    \item \cmakebox{STATIC} - Value managed by CMake, do not change
    \item \cmakebox{UNINITIALIZED} - Type not yet specified
\end{itemize}



\subsection{Lists}
For the following and many other commands, see \href{https://cmake.org/cmake/help/latest/command/list.html}{CMake documentation}.

\subsubsection{Populating lists}
Lists are first defined via \cmakebox{set(<list_name>) ["<value1>" "<value2>" ...]} and later increased via \cmakebox{list(APPEND <list_name> "<value>")}. \\
Alternatively, for filenames, use \cmakebox{file([GLOB | GLOB_RECURSE] <list_name> <pattern1> <pattern2> ...)}, e.g. \cmakebox{file(GLOB_RECURSE <list_name> src/*.cpp)}.

\subsubsection{Using lists}
To access the elements of a list, use \cmakebox{${<list_name>_<index>}}, possibly within a for loop (Section \ref{ss:conditionals/for_loop}). \\
Lists of filenames can be directly used in the \cmakebox{add_library()} or \cmakebox{add_executable()} commands, e.g. \cmakebox{add_library(${PROJECT_NAME} <list_name>)}.



\subsection{Set Variables}

\subsubsection{\texttt{option}}
\begin{center} \cmakebox{option(<variable> "<help text>" <value>)} \end{center}
Defines a boolean variable as \cmakebox{ON} or \cmakebox{OFF} (default). If already defined, it will NOT be changed.

\subsubsection{\texttt{set}}
\begin{center} \cmakebox{set(<variable> [<value>])} \end{center}
Assign a value to a variable. If \cmakebox{<value>} is not provided, the variable is unset (set to an empty string; equivalent to \cmakebox{unset(<variable>)}).

\subsubsection{\texttt{set}}
\begin{center} \cmakebox{set(<variable> <value>... CACHE <type> <docstring> [FORCE])} \end{center}
Set a variable to a value and store it in the cache (i.e. is not re-built). \cmakebox{<type>} is one of the base types (Section \ref{ss:variables/base_types}). The \cmakebox{FORCE} option will overwrite the variable in the cache even if it is already defined.

Example: \cmakebox{set(TEST "Hello, World!" CACHE STRING "Test variable")}




% TODO: Following 2 commands

% \subsubsection{\texttt{set\_property}} % for exact GLOBAL/TARGET/... details see https://cmake.org/cmake/help/latest/command/set_property.html#command:set_property
% {\centering\cmakebox{set_property([GLOBAL | TARGET <target>] PROPERTY <prop> <value>)}} \\
% Most (if not all) should simply be settable using \cmakebox{set}. \\
% Existing properties are:
% \begin{itemize}
%     \item \cmakebox{abc} - description
%     \item \ldots
% \end{itemize}
% GLOBAL:
% CMAKE_CXX_STANDARD: Sets the global C++ standard version for the entire project.
% CMAKE_C_FLAGS: Global C compiler flags.
% CMAKE_CXX_FLAGS: Global C++ compiler flags.
% CMAKE_INSTALL_PREFIX: Defines the default installation path.
% TARGET:
% SOURCES: Adds source files to the target.
% INCLUDE_DIRECTORIES: Adds include directories for the target.
% COMPILE_OPTIONS: Adds compile options to the target.
% LINK_FLAGS: Adds linker flags for a target.
% DIRECTORY:
% CMAKE_BUILD_TYPE: Sets the build type (e.g., Debug, Release).
% CMAKE_INCLUDE_PATH: Sets the path to search for CMake include files.
% CMAKE_LIBRARY_PATH: Sets the path to search for libraries.
% SOURCE:
% HEADER_FILE_ONLY: Marks a source file as a header-only file.
% FILE:
% FLAGS: Sets the properties of a file (e.g., source file properties).

% \subsubsection{\texttt{set\_target\_properties}}
% {\centering\cmakebox{set_target_properties(<target> PROPERTIES <prop> <value>)}}
% Existing target properties are:
% \begin{itemize}
%     \item \cmakebox{abc} - description
%     \item \ldots
% \end{itemize}
% ARCHIVE_OUTPUT_DIRECTORY: Specifies the output directory for static libraries.
% LIBRARY_OUTPUT_DIRECTORY: Specifies the output directory for shared libraries.
% RUNTIME_OUTPUT_DIRECTORY: Specifies the output directory for executables.
% LINKER_LANGUAGE: Specifies the language to be used by the linker (e.g., CXX, C, Fortran).
% COMPILE_DEFINITIONS: Adds preprocessor definitions to a target.
% COMPILE_OPTIONS: Adds compile flags/options for a target.
% INCLUDE_DIRECTORIES: Specifies directories to be added to the target's include search path.
% INTERFACE_INCLUDE_DIRECTORIES: Specifies include directories that are used by the target’s consumers.
% PUBLIC/PRIVATE/INTERFACE: These are used to define whether a property is used only by the target (PRIVATE), is used by the target and also exposed to consumers (PUBLIC), or only exposed to consumers (INTERFACE).
% CXX_STANDARD: Specifies the C++ standard to be used for the target (e.g., 11, 14, 17).
% OUTPUT_NAME: Defines the name of the output binary.
% VERSION: Specifies the version number of the target (commonly used for libraries).
% SOVERSION: Specifies the shared library version (useful for shared libraries).
% FOLDER: Specifies the folder that the target will appear in within IDEs (e.g., Visual Studio).
% WIN32_EXECUTABLE: Specifies if a target is a Windows executable.
% POSITION_INDEPENDENT_CODE: If set to true, enables the generation of position-independent code (e.g., for shared libraries).
% CXX_FLAGS: Adds flags specific to the C++ compiler.
% C_FLAGS: Adds flags specific to the C compiler.
% CUDA_ARCHITECTURES: Defines the CUDA architecture for a target (if CUDA is used).